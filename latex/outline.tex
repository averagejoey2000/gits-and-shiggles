\documentclass[oneside,12pt,letterpaper]{article}
\author{Joseph Simone}
\title{Three Words or Fewer}
\today
%\usepackage{syntaxonly}
%\syntaxonly
\begin{document}
This Document Was Produced with \latex
\maketitle
\\
\tableofcontents
\\
The Point you are trying to make in a simple sentence
\section[My First Point]{Why X is Y}
Words
\subsection[Point 1.1]{Backing up the unifying idea}
Slightly more complex thought that hones in on the true meaning of the paragraph
\subsection[Point 1.2]{A counterpoint to 1.1 to fairly show both sides of the argument}
\subsubsection[1.2.A]{Why 1.2 is wrong}
Evidence \cite{evid} That proves that 1.2 is wrong, and therefore 1.1 is right. \\

Paragraph Conclusion
\\
\\
\section[My Second Point]{Why Y is Z}
Words
\subsection[Point 2.1]{Backing up the unifying idea}
Slightly more complex thought that hones in on the true meaning of the paragraph
\subsection[Point 2.2]{A counterpoint to 1.1 to fairly show both sides of the argument}
\subsubsection[2.2.A]{Why 1.2 is wrong}
Paragraph Conclusion

\section[Call to Action]{Because X is Z We Must W}
Basic Premise Slightly longer than the Thesis Statement
\subsection{What to Do}
\subsubsection{How to do it}
\subsection{Why They Should}

Heartfelt and inspiring rehashing of the thesis statement. %The Thing Before Section one
But delivered in more than one complex sentence, with emotion. %Emotion optional, but important

\begin{thebibliography}

\bibitem{evid} Why X is Y by \\ Accomplished Researcher; Page 172; ISBN:48202-24521-432

\end{thebibliography}
\end{document}
